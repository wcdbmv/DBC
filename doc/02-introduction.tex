\chapter*{Введение}
\addcontentsline{toc}{chapter}{Введение}

В современном мире, одну из лидирующих позиций занимает информационное пространство.
Это публичная площадка в сети Интернет, где человек излагает свои мысли.
Множество людей наблюдает за подобными публичными площадками и людьми, которые их ведут в различных социальных сетях.

Один из наиболее интересных видов информационного пространства~— блог.
Термин «блог» произошёл от английского weblog («logging the web»~— записывать события в сеть).
Впервые его использовал американский программист Йорн Баргер в 1997 году для обозначения сетевого дневника \cite{lagoshina}.

\textbf{Целью} данного курсового проекта является создание платформы для ведения онлайн-дневников (блогов).

\section*{Задачи работы}

В рамках выполнения проекта необходимо решить следующие задачи:
\begin{itemize}
	\item формализовать задачу в виде определения необходимого функционала;
	\item провести анализ существующих СУБД;
	\item спроектировать базу данных, необходимую для хранения и структурирования данных;
	\item программно реализовать спроектированную базу данных с использованием выбранной СУБД;
	\item разработать приложение для взаимодействия с реализованной БД.
\end{itemize}
