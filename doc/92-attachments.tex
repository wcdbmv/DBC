\chapter*{Приложение А. Сгенерированная структура базы данных}
\addcontentsline{toc}{chapter}{Приложение А. Сгенерированная структура базы данных}
\label{chp:attachment-a}

\begin{lstlisting}[caption={Сгенерированная структура базы данных}, label={lst:sqlmigrate}, language=SQL]
BEGIN;

--
-- Create model Article
--
CREATE TABLE "myblog_article" (
	"id" serial NOT NULL PRIMARY KEY,
	"title" varchar(80) NOT NULL,
	"body" text NOT NULL,
	"pub_date" timestamp with time zone NOT NULL,
	"rating" smallint NOT NULL
);

--
-- Create model Tag
--
CREATE TABLE "myblog_tag" (
	"id" serial NOT NULL PRIMARY KEY,
	"tag" varchar(50) NOT NULL UNIQUE
);

--
-- Create model Vote
--
CREATE TABLE "myblog_vote" (
	"id" serial NOT NULL PRIMARY KEY,
	"value" smallint NOT NULL,
	"object_id" integer NOT NULL CHECK ("object_id" >= 0),
	"content_type_id" integer NOT NULL,
	"user_id" integer NOT NULL
);

--
-- Create model Comment
--
CREATE TABLE "myblog_comment" (
	"id" serial NOT NULL PRIMARY KEY,
	"body" text NOT NULL,
	"pub_date" timestamp with time zone NOT NULL,
	"rating" smallint NOT NULL,
	"article_id" integer NOT NULL,
	"user_id" integer NOT NULL
);

--
-- Add field tags to article
--
CREATE TABLE "myblog_article_tags" (
	"id" serial NOT NULL PRIMARY KEY,
	"article_id" integer NOT NULL,
	"tag_id" integer NOT NULL
);

--
-- Add field user to article
--
ALTER TABLE "myblog_article"
	ADD COLUMN "user_id" integer NOT NULL
	CONSTRAINT "myblog_article_user_id_7728319b_fk_auth_user_id"
	REFERENCES "auth_user"("id") DEFERRABLE INITIALLY DEFERRED;
SET CONSTRAINTS "myblog_article_user_id_7728319b_fk_auth_user_id" IMMEDIATE;

CREATE INDEX "myblog_tag_tag_e004ad11_like" ON "myblog_tag" ("tag" varchar_pattern_ops);

ALTER TABLE "myblog_vote"
	ADD CONSTRAINT "myblog_vote_content_type_id_06bf44c6_fk_django_content_type_id"
	FOREIGN KEY ("content_type_id") REFERENCES "django_content_type" ("id")
	DEFERRABLE INITIALLY DEFERRED;

ALTER TABLE "myblog_vote"
	ADD CONSTRAINT "myblog_vote_user_id_89a11e40_fk_auth_user_id"
	FOREIGN KEY ("user_id") REFERENCES "auth_user" ("id") DEFERRABLE INITIALLY DEFERRED;

CREATE INDEX "myblog_vote_content_type_id_06bf44c6" ON "myblog_vote" ("content_type_id");

CREATE INDEX "myblog_vote_user_id_89a11e40" ON "myblog_vote" ("user_id");

ALTER TABLE "myblog_comment"
	ADD CONSTRAINT "myblog_comment_article_id_44b1452c_fk_myblog_article_id"
	FOREIGN KEY ("article_id") REFERENCES "myblog_article" ("id")
	DEFERRABLE INITIALLY DEFERRED;

ALTER TABLE "myblog_comment"
	ADD CONSTRAINT "myblog_comment_user_id_1d5be68b_fk_auth_user_id"
	FOREIGN KEY ("user_id") REFERENCES "auth_user" ("id") DEFERRABLE INITIALLY DEFERRED;

CREATE INDEX "myblog_comment_article_id_44b1452c" ON "myblog_comment" ("article_id");

CREATE INDEX "myblog_comment_user_id_1d5be68b" ON "myblog_comment" ("user_id");

ALTER TABLE "myblog_article_tags"
	ADD CONSTRAINT "myblog_article_tags_article_id_tag_id_7e4e85c7_uniq"
	UNIQUE ("article_id", "tag_id");

ALTER TABLE "myblog_article_tags"
	ADD CONSTRAINT "myblog_article_tags_article_id_04a3a3b7_fk_myblog_article_id"
	FOREIGN KEY ("article_id") REFERENCES "myblog_article" ("id")
	DEFERRABLE INITIALLY DEFERRED;

ALTER TABLE "myblog_article_tags"
	ADD CONSTRAINT "myblog_article_tags_tag_id_10cdc0fe_fk_myblog_tag_id"
	FOREIGN KEY ("tag_id") REFERENCES "myblog_tag" ("id") DEFERRABLE INITIALLY DEFERRED;

CREATE INDEX "myblog_article_tags_article_id_04a3a3b7" ON "myblog_article_tags" ("article_id");

CREATE INDEX "myblog_article_tags_tag_id_10cdc0fe" ON "myblog_article_tags" ("tag_id");

CREATE INDEX "myblog_article_user_id_7728319b" ON "myblog_article" ("user_id");

COMMIT;
\end{lstlisting}

\chapter*{Приложение Б. Представления}
\addcontentsline{toc}{chapter}{Приложение Б. Представления}
\label{chp:attachment-b}

В листингах \ref{lst:ArticleView}–\ref{lst:ArticleDelete} показаны представления, связанные со страницей статьи.
Используются стандартные представления \code{DetailView}, \code{CreateView}, \code{UpdateView}, \code{DeleteView} из модуля \code{django.views.generic}.

\lstinputlisting[
	caption={Представление страницы статьи},
	label={lst:ArticleView},
	language=Python,
	linerange={10-24}
]{../myblog/views/article.py}

\lstinputlisting[
	caption={Представление страницы создания статьи},
	label={lst:ArticleCreate},
	language=Python,
	linerange={27-35}
]{../myblog/views/article.py}

\lstinputlisting[
	caption={Представление страницы редактирования статьи},
	label={lst:ArticleUpdate},
	language=Python,
	linerange={38-42}
]{../myblog/views/article.py}

\lstinputlisting[
	caption={Представление страницы удаления статьи},
	label={lst:ArticleDelete},
	language=Python,
	linerange={45-49}
]{../myblog/views/article.py}

В листингах \ref{lst:FeedView}–\ref{lst:TagView} показаны представления списка статей, основанные на стандартном представлении \code{ListView} из модуля \code{django.views.generic}.

\lstinputlisting[
	caption={Представление страницы ленты статей},
	label={lst:FeedView},
	language=Python,
	linerange={14-20}
]{../myblog/views/article_list.py}

\lstinputlisting[
	caption={Представление страницы статей определённого пользователя},
	label={lst:BlogView},
	language=Python,
	linerange={23-42}
]{../myblog/views/article_list.py}

\lstinputlisting[
	caption={Представление страницы статей определённого тэга},
	label={lst:TagView},
	language=Python,
	linerange={45-59}
]{../myblog/views/article_list.py}

В листингах \ref{lst:CommentCreate}–\ref{lst:CommentDelete} показаны представления, связанные с комментариями.
Используются стандартные представления \code{DetailView}, \code{CreateView}, \code{UpdateView}, \code{DeleteView} из модуля \code{django.views.generic}.

В листингах \ref{lst:CommentCreate} и \ref{lst:RegisterView} показаны представления страницы создания комментария и регистрации соответственно.
Используются стандартное представление \code{CreateView} из модуля \code{django.views.generic}.

\lstinputlisting[
	caption={Представление страницы создания комментария},
	label={lst:CommentCreate},
	language=Python,
	linerange={9-21}
]{../myblog/views/comment.py}

\lstinputlisting[
	caption={Представление страницы обновления комментария},
	label={lst:CommentUpdate},
	language=Python,
	linerange={24-31}
]{../myblog/views/comment.py}

\lstinputlisting[
	caption={Представление страницы удаления комментария},
	label={lst:CommentDelete},
	language=Python,
	linerange={34-40}
]{../myblog/views/comment.py}

\lstinputlisting[
	caption={Представление страницы регистрации},
	label={lst:RegisterView},
	language=Python,
	linerange={7-16}
]{../myblog/views/register.py}

\chapter*{Приложение В. Шаблоны}
\addcontentsline{toc}{chapter}{Приложение В. Шаблоны}
\label{chp:attachment-c}

\lstinputlisting[
	caption={Шаблон основы страницы},
	label={lst:template-base},
	language=HTML
]{../myblog/templates/myblog/base.html}

\lstinputlisting[
	caption={Шаблон страниц списка статей},
	label={lst:template-article-list},
	language=HTML
]{../myblog/templates/myblog/article_list.html}

\lstinputlisting[
	caption={Шаблон пагинации},
	label={lst:template-pagination},
	language=HTML
]{../myblog/templates/myblog/pagination.html}

\lstinputlisting[
	caption={Шаблон страницы логина},
	label={lst:template-login},
	language=HTML
]{../myblog/templates/myblog/login.html}

\lstinputlisting[
	caption={Шаблон страницы регистрации},
	label={lst:template-register},
	language=HTML
]{../myblog/templates/myblog/register.html}

\lstinputlisting[
	caption={Шаблон страницы создания},
	label={lst:template-create},
	language=HTML
]{../myblog/templates/myblog/create.html}

\lstinputlisting[
	caption={Шаблон страницы подтверждения удаления},
	label={lst:template-confirm-delete},
	language=HTML
]{../myblog/templates/myblog/confirm_delete.html}

\chapter*{Приложение Г. Генерация фейковых данных}
\addcontentsline{toc}{chapter}{Приложение Г. Генерация фейковых данных}
\label{chp:attachment-d}

\lstinputlisting[
	caption={Генерация фейковых данных},
	label={lst:generate},
	language=Python,
]{../myblog/management/commands/generate.py}
