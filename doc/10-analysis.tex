\chapter{Аналитический раздел}

\section{Разбор различных СУБД}

Несмотря на то, что все системы управления базами данных выполняют одну и ту же основную задачу (т.~е. дают возможность пользователям создавать, редактировать и получать доступ к информации, хранящейся в базах данных), сам процесс выполнения этой задачи варьируется в широких пределах.
Кроме того, функции и возможности каждой СУБД могут существенно отличаться.
Различные СУБД документированы по-разному: более или менее тщательно.
По-разному предоставляется и техническая поддержка.
При сравнении различных популярных баз данных, следует учитывать, удобна ли для пользователя и масштабируема ли данная конкретная СУБД, а также убедиться, что она будет хорошо интегрироваться с другими продуктами, которые будут использоваться в системе.
Для разработки была выбрана реляционная модель базы данных, поэтому рассмотрим три наиболее важные и популярные СУБД с открытым исходным кодом: SQLite, MySQL, PostgreSQL.
SQLite — это производительная библиотека, которую можно встраивать в приложения, полноценная БД на основе файлов SQLite предлагает широкий набор инструментов, поэтому она и будет выбрана для реализации проекта.

\subsection{SQLite}

База данных SQLite накладывает минимальное количество ограничений на пользователя.
База не предоставляет сетевого интерфейса и использует для своей работы единственный файл.
Первоначально была разработана как облегченная версия MySQL без модулей, которые не используются во многих проектах.

Преимущества:
\begin{itemize}
	\item Главное преимущество заключается в том, что во встраиваемых или распределенных системах каждая машина несет полную реализацию базы данных.
	Это может значительно повысить производительность БД, поскольку снижает потребность в межпроцессных вызовах.\\
	Например, база встроена в ОС Android и предоставляет разработчикам приложений удобный интерфейс работы с ней.
	\item База использует для работы единственный файл, что хорошо влияет на ее переносимость.
	\item Система SQLite основана на языке запросов SQL, что удобно для разработки.
	\item Подходит для небольших проектов с небольшими БД.
\end{itemize}

Недостатки:
\begin{itemize}
	\item Отсутствие встроенного шифрования данных, что стало стандартом для предотвращения наиболее распространенных хакерских атак в интернете.
	\item База не поддерживает систему учета пользователей, в то время как другие популярные СУБД поддерживают эту возможность.
	\item Только одна операция записи за транзакцию, что уменьшает производительность системы
\end{itemize}

\subsection{MySQL}

MySQL — самая популярная СУБД.
Это многофункциональное открытое приложение, поддерживающее работу огромного количества сайтов.
Система MySQL довольно проста в работе и может хранить большие массивы данных.
Учитывая популярность MySQL, для этой системы было разработано большое количество сторонних приложений, инструментов и библиотек.
MySQL не реализует полный стандарт SQL.
Несмотря на это, MySQL предлагает множество функциональных возможностей для пользователей: автономный сервер баз данных, взаимодействие с приложениями и сайтами и т.~п.

Преимущества:
\begin{itemize}
	\item Простота в работе: MySQL очень просто установить и настроить.
	Сторонние инструменты, в том числе визуализаторы (интерфейсы) значительно упрощают работу с данными.
	\item Функциональность: MySQL поддерживает огромное количество функций SQL.
	\item Безопасность: MySQL предоставляет много встроенных продвинутых функций для защиты данных.
	\item Масштабируемость и производительность: MySQL может работать с большими объёмами данных.
\end{itemize}

Недостатки:
\begin{itemize}
	\item Ограничения: структура MySQL накладывает некоторые ограничения, из‒за которых не смогут работать продвинутые приложения.
	\item Уязвимости: метод обработки данных, применяемый в MySQL, делает эту СУБД немного менее надёжной по сравнению с другими СУБД.
	\item Медленное развитие: хотя MySQL является продуктом с открытым исходным кодом, он очень медленно развивается.
	Однако тут следует заметить, что на MySQL основано несколько полноценных баз данных (например, MariaDB).
\end{itemize}

\subsection{PostgreSQL}

PostgreSQL — это продвинутая открытая объектно-ориентированная СУБД.
Часто используется при разработке веб-сайтов.
Позволяет разработчикам управлять как структурированными, так и неструктурированными данными.
Может быть использована на большинстве основных платформ, включая Linux и MacOS.
В отличие от других СУБД, PostgreSQL поддерживает очень важные объектно‒ориентированные и реляционные функции баз данных: надежные транзакции ACID (атомарность, согласованность, изолированность, долговечность) и т.~п.
СУБД PostgreSQL основана на надежной технологии, она может одновременно обрабатывать большое количество задач.
Хотя СУБД PostgreSQL не так популярна, как MySQL, для неё тоже разработано большое количество дополнительных инструментов и библиотек, которые упрощают работу с данными и увеличивают производительность СУБД.

Преимущества:
\begin{itemize}
	\item PostgreSQL имеет открытый исходный код.
	\item База использует язык запросов SQL.
	\item PostgreSQL обладает большим сообществом разработчиков, которые помогут найти решение любой проблемы, связанной с СУБД, в любое время суток.
	\item Помимо встроенных функций, PostgreSQL поддерживает множество открытых сторонних инструментов для проектирования, управления данными и т.~п. Одним из таких инструментов является веб-приложение pgAdmin.
	\item База очень хорошо масштабируется и расширяется.
	\item Работает быстро и надежно, база способна обрабатывать терабайты данных.
	\item Поддерживает формат json.
\end{itemize}

Недостатки:
\begin{itemize}
	\item Производительность в некоторых ситуациях ниже, чем у MySQL.
	\item Невысокая популярность, однако все больше проектов используют PostgreSQL в качестве СУБД.
\end{itemize}

\section{Django}

Django — фреймворк для веб-приложений на языке Python.
Один из основных принципов фреймворка — DRY (don't repeat yourself).
Веб-системы на Django строятся из одного или нескольких приложений, которые рекомендуется делать отчуждаемыми и подключаемыми.
Это одно из заметных архитектурных отличий этого фреймворка от некоторых других, например, Ruby on Rails.
Также, в отличие от многих других фреймворков, обработчики URL в Django конфигурируются явно при помощи регулярных выражений, а не автоматически задаются из структуры контроллеров.

Django проектировался для работы под управлением Apache с модулем mod\_python и с использованием PostgreSQL в качестве базы данных.
В настоящее время, помимо PostgreSQL, Django может работать с другими СУБД: MySQL, SQLite, Microsoft SQL Server, DB2, Firebird, SQL Anywhere 9 и Oracle.
Для работы с базой данных Django использует собственный ORM, в котором модель данных описывается классами Python, и по ней генерируется схема базы данных.

Архитектура Django похожа на MVC (Model-View-Controller).
Контроллер классической модели MVC примерно соответствует уровню, который в Django называется Представлением (View), а презентационная логика Представления реализуется в Django уровнем Шаблонов (Templates).
Из-за этого уровневую архитектуру Django часто называют MTV (Model-Template-View).

Первоначально разработка Django велась для обеспечения более удобной работы с новостными ресурсами, что достаточно сильно отразилось на архитектуре: фреймворк предоставляет ряд средств, которые помогают в быстрой разработке веб-сайтов информационного характера.
Например, разработчику не требуется создавать контроллеры и страницы для административной части сайта, в Django есть встроенное приложение для управления содержимым, которое можно включить в любой сайт, сделанный на Django, и которое может управлять сразу несколькими сайтами на одном сервере.
Административное приложение позволяет создавать, изменять и удалять любые объекты наполнения сайта, протоколируя все совершённые действия, и предоставляет интерфейс для управления пользователями и группами.

Веб-фреймворк Django используется в таких крупных и известных сайтах, как Instagram, Disqus, Mozilla, The Washington Times, Pinterest, lamoda.

Преимущества:
\begin{itemize}
	\item Простота в изучении.
	\item Чистота и читаемость.
	\item Разносторонность.
	\item Быстрота написания.
	\item Цельный дизайн
\end{itemize}
