\chapter{Аналитический раздел}

В данном разделе будут рассмотрены общие сведения о БД и СУБД, популярные СУБД и используемый фреймворк.

\section{Формализация задачи}

В соответствии с техническим заданием на курсовой проект необходимо разработать платформу для ведения онлайн-дневников (блогов), в которой пользователи смогут авторизоваться и аутентифицироваться, писать статьи и комментарии к ним.
Необходимо обеспечить возможность пользователей оценивать статьи и комментарии, сортировать их по дате публикации и по рейтингу, просматривать список статей определённого автора, а также фильтровать их по тэгам.

\section{Общие сведения о БД и СУБД}

База данных — представленная в объективной форме совокупность самостоятельных материалов (статей, расчётов, нормативных актов, судебных решений и иных подобных материалов), систематизированных таким образом, чтобы эти материалы могли быть найдены и обработаны с помощью электронной вычислительной машины (ЭВМ) \cite{civil}.

Под системой управления базами данных (СУБД) понимается совокупность программных и лингвистических средств общего или специального назначения, обеспечивающих управление созданием и использованием баз данных \cite{gost2007}.

\section{Разбор различных СУБД}

Несмотря на то, что все системы управления базами данных выполняют одну и ту же основную задачу, сам процесс выполнения этой задачи варьируется в широких пределах.
Кроме того, функции и возможности каждой СУБД могут существенно отличаться.
Различные СУБД документированы по-разному: более или менее тщательно.
По-разному предоставляется и техническая поддержка.
При сравнении различных популярных баз данных, следует учитывать, удобна ли для пользователя и масштабируема ли данная конкретная СУБД, а также убедиться, что она будет хорошо интегрироваться с другими продуктами, которые будут использоваться в системе.
Для разработки была выбрана реляционная модель базы данных, поэтому рассмотрим три наиболее важные и популярные СУБД с открытым исходным кодом: SQLite, MySQL, PostgreSQL.

\subsection{SQLite}

База данных SQLite накладывает минимальное количество ограничений на пользователя.
База не предоставляет сетевого интерфейса и использует для своей работы единственный файл.
Первоначально была разработана как облегченная версия MySQL без модулей, которые не используются во многих проектах \cite{sqlite}.

Преимущества:
\begin{itemize}
	\item[$+$] Главное преимущество заключается в том, что во встраиваемых или распределенных системах каждая машина несет полную реализацию базы данных.
	Это может значительно повысить производительность БД, поскольку снижает потребность в межпроцессных вызовах.
	Например, база встроена в ОС Android и предоставляет разработчикам приложений удобный интерфейс работы с ней.
	\item[$+$] База использует для работы единственный файл, что хорошо влияет на ее переносимость.
	\item[$+$] Подходит для небольших проектов с небольшими БД.
\end{itemize}

Недостатки:
\begin{itemize}
	\item[$-$] Отсутствие встроенного шифрования данных, что стало стандартом для предотвращения наиболее распространенных хакерских атак в интернете.
	\item[$-$] База не поддерживает систему учета пользователей, в то время как другие популярные СУБД поддерживают эту возможность.
	\item[$-$] Только одна операция записи за транзакцию, что уменьшает производительность системы
\end{itemize}

\subsection{MySQL}

MySQL — самая популярная СУБД \cite{mysql}.
Это многофункциональное открытое приложение, поддерживающее работу огромного количества сайтов.
Система MySQL довольно проста в работе и может хранить большие массивы данных.
Учитывая популярность MySQL, для этой системы было разработано большое количество сторонних приложений, инструментов и библиотек.
MySQL не реализует полный стандарт SQL.
Несмотря на это, MySQL предлагает множество функциональных возможностей для пользователей: автономный сервер баз данных, взаимодействие с приложениями и сайтами и т.~п.

\textbf{Преимущества:}
\begin{itemize}
	\item[$+$] Простота в работе: MySQL очень просто установить и настроить.
	Сторонние инструменты, в том числе визуализаторы (интерфейсы) значительно упрощают работу с данными.
	\item[$+$] Безопасность: MySQL предоставляет много встроенных продвинутых функций для защиты данных.
	\item[$+$] Масштабируемость и производительность: MySQL может работать с большими объёмами данных.
\end{itemize}

\textbf{Недостатки:}
\begin{itemize}
	\item[$-$] Ограничения: структура MySQL накладывает некоторые ограничения, из-за которых не смогут работать продвинутые приложения.
	\item[$-$] Уязвимости: метод обработки данных, применяемый в MySQL, делает эту СУБД немного менее надёжной по сравнению с другими СУБД.
	\item[$-$] Медленное развитие: хотя MySQL является продуктом с открытым исходным кодом, он очень медленно развивается.
	Однако тут следует заметить, что на MySQL основано несколько полноценных баз данных (например, MariaDB).
\end{itemize}

\subsection{PostgreSQL}

PostgreSQL — это продвинутая открытая объектно-ориентированная СУБД.
Часто используется при разработке веб-сайтов.
Позволяет разработчикам управлять как структурированными, так и неструктурированными данными.
Может быть использована на большинстве основных платформ, включая Linux и MacOS.
В отличие от других СУБД, PostgreSQL поддерживает очень важные объектно-ориентированные и реляционные функции баз данных: надежные транзакции ACID (атомарность, согласованность, изолированность, долговечность) \cite{postgresql} и т.~п.
СУБД PostgreSQL основана на надежной технологии, она может одновременно обрабатывать большое количество задач.
Хотя СУБД PostgreSQL не так популярна, как MySQL, для неё тоже разработано большое количество дополнительных инструментов и библиотек, которые упрощают работу с данными и увеличивают производительность СУБД.

\textbf{Преимущества:}
\begin{itemize}
	\item[$+$] PostgreSQL имеет открытый исходный код.
	\item[$+$] PostgreSQL обладает большим сообществом разработчиков, которые помогут найти решение любой проблемы, связанной с СУБД, в любое время суток.
	\item[$+$] Помимо встроенных функций, PostgreSQL поддерживает множество открытых сторонних инструментов для проектирования, управления данными и т.~п.
	Одним из таких инструментов является веб-приложение pgAdmin.
	\item[$+$] База очень хорошо масштабируется и расширяется.
	\item[$+$] Работает быстро и надежно, база способна обрабатывать терабайты данных.
	\item[$+$] Поддерживает формат json.
\end{itemize}

\textbf{Недостатки:}
\begin{itemize}
	\item[$-$] Производительность в некоторых ситуациях ниже, чем у MySQL.
	\item[$-$] Невысокая популярность, однако все больше проектов используют PostgreSQL в качестве СУБД.
\end{itemize}

В данном курсовом проекте будет использоваться PostgreSQL ввиду объективного превосходства над всеми остальными СУБД.

\section{Django}

Django — фреймворк для веб-приложений на языке Python.
Один из основных принципов фреймворка — DRY (don't repeat yourself).
Веб-системы на Django строятся из одного или нескольких приложений, которые рекомендуется делать отчуждаемыми и подключаемыми.
Это одно из заметных архитектурных отличий этого фреймворка от некоторых других, например, Ruby on Rails.
Также, в отличие от многих других фреймворков, обработчики URL в Django конфигурируются явно при помощи регулярных выражений, а не автоматически задаются из структуры контроллеров.

Django проектировался для работы под управлением Apache с модулем mod\_python и с использованием PostgreSQL в качестве базы данных.
В настоящее время, помимо PostgreSQL, Django может работать с другими СУБД: MySQL, SQLite, Microsoft SQL Server, DB2, Firebird, SQL Anywhere~9 и Oracle.
Для работы с базой данных Django использует собственный ORM, в котором модель данных описывается классами Python, и по ней генерируется схема базы данных \cite{saglaev}.

Архитектура Django похожа на MVC (Model-View-Controller).
Контроллер классической модели MVC примерно соответствует уровню, который в Django называется Представлением (View), а презентационная логика Представления реализуется в Django уровнем Шаблонов (Templates).
Из-за этого уровневую архитектуру Django часто называют MTV (Model-Template-View).

Первоначально разработка Django велась для обеспечения более удобной работы с новостными ресурсами, что достаточно сильно отразилось на архитектуре: фреймворк предоставляет ряд средств, которые помогают в быстрой разработке веб-сайтов информационного характера.
Например, разработчику не требуется создавать контроллеры и страницы для административной части сайта, в Django есть встроенное приложение для управления содержимым, которое можно включить в любой сайт, сделанный на Django, и которое может управлять сразу несколькими сайтами на одном сервере.
Административное приложение позволяет создавать, изменять и удалять любые объекты наполнения сайта, протоколируя все совершённые действия, и предоставляет интерфейс для управления пользователями и группами.

Веб-фреймворк Django используется на таких крупных и известных сайтах, как Instagram, Disqus, Mozilla, The Washington Times, Pinterest, lamoda.

\textbf{Преимущества:}
\begin{itemize}
	\item[$+$] Простота в изучении.
	\item[$+$] Чистота и читаемость.
	\item[$+$] Разносторонность.
	\item[$+$] Быстрота написания.
	\item[$+$] Цельный дизайн.
\end{itemize}

\section*{Вывод}

В результате проведённого анализа в качестве СУБД выбран PostgreSQL.
Фреймворком для разрабатываемого веб-приложения выбран Django как наиболее комфортный.
